\documentclass{beamer}
\usepackage{../../format}

\title{Lecture 1}
\subtitle{Important Concepts In Set Theory}
\author{}
\date{\today}

\begin{document}
		\maketitle

		\begin{frame}{Sets}
				\begin{definition}[Sets]
						Put simply, a \alert{set} is simply a collection of objects. Namely, in sets
						\begin{itemize}
								\item The order of items does not matter 
								\item The number of occurences of an item does not matter
						\end{itemize}
				\end{definition}
				For example, 
				\begin{itemize}
						\item As sets, $\{1, 2, 3\}$ is the same set as $\{3, 1, 2\}$ and as $\{2, 1, 3\}$ since \emph{order does not matter}
						\item Also, $\{1, 1, 2, 3\}$ is the same set as $\{1, 2, 3\}$ since the \emph{number of occurences does not matter}
				\end{itemize}
		\end{frame}

		\begin{frame}{Representing Sets}
				How do we write out sets? 
				\begin{itemize}
						\item If the set is finite, we may simply list out its elements 
								\begin{equation*}
										\{1, 2, 3, 4, 5, 6\}
								\end{equation*}
						\item We may also do the same thing if the set is infinite 
								\begin{equation*}
										\{1, 2, 3, 4, 5, ...\}
								\end{equation*}
				\end{itemize}
				But the latter risks being ambigious in certain situations. 
		\end{frame}

		\begin{frame}{Representing Sets}
				\footnotesize
				For infinite sets, we may avoid ambiguity by using \alert{set-builder notation}. 
				\begin{definition}[Set-Builder Notation (Parametric Form)]
						We may write a set as 
						\begin{equation*}
								\{f(t) : t\in T\}
						\end{equation*}
						Here $f$, is some expression which depends on $t$ while $T$ is a set that contains all possible values of $t$. 
				\end{definition}
				\begin{definition}[Set-Builder Notation (Conditional Form)]
						Another way to use set builder notation is as 
						\begin{equation*}
								\{t\in T : P(t)\}
						\end{equation*}
						Here, $t$ and $T$ are the same as before, but now $P$ is an \emph{open sentence} in $t$. The set contains only the values of $t$ which make $P(t)$ true. 
				\end{definition}
				An \alert{open sentence} in $t$ is just a statement whose truth value depends on the value of $t$. 
		\end{frame}

		\begin{frame}{Special Sets}
				Some sets are so common in math, they are given special symbols 
				\begin{itemize}
						\item $\N$: the set of \alert{natural numbers}
								\begin{equation*}
										\N = \{1, 2, 3, 4, 5, ...\}
								\end{equation*}
						\item $\Z$: the set of \alert{integers}
								\begin{equation*}
										\Z = \{..., -3, -2, -1, 0, 1, 2, 3, ...\}
								\end{equation*}
						\item $\Q$: the set of \alert{rational numbers}
								\begin{equation*}
										\Q = \left\{ \frac{p}{q} : p, q\in \Z, q \neq 0\right\}
								\end{equation*}
						\item $\R$: the set of \alert{real numbers}
				\end{itemize}
		\end{frame}

		\begin{frame}{Basic Notation \& Terminology About Sets}
				\begin{definition}[]
						Here are some basic definitions regarding sets 
						\begin{itemize}
								\item A set need not contain any elements. The \alert{empty set} is the unique set which contains no elements. It is often denoted by $\emptyset$ (commonly) or by $\{\}$ (not really common)
								\item A set which contains at least one element is said to be \alert{nonempty}. 
								\item If $A$ is nonempty and $x$ is an element of $A$, we say ``$x$ is a member of $A$'' and write $x\in A$. 
										\begin{itemize}
												\item We may also write this as $A \ni x$
										\end{itemize}
								\item Likewise, $x\not\in A$ means ``$x$ is not a member of $A$'' 
										\begin{itemize}
												\item We may also write $A \not \ni x$
										\end{itemize}
						\end{itemize}
				\end{definition}
		\end{frame}

		\begin{frame}{Examples}
				\begin{example}[]
						What is the set $E$ defined as $E = \{2k : k\in \Z\}$? 
				\end{example}
				\begin{itemize}
						\item We have that $0\in E$, $2\in E$, and $-110\in E$ for example 
						\item However, $1\not\in E$, $-19\not\in E$, and $9\not\in E$
				\end{itemize}

				\begin{example}[]
						What is the set $T$ defined as 
						\begin{equation*}
								T = \left\{x\in\R : x \neq \frac{a}{b} \text{ for all integers $a$ and $b$}, b \neq 0\right\}
						\end{equation*}
				\end{example}
				\begin{itemize}
						\item We have $\pi \in T$ and $\sqrt{2}\in T$ 
						\item But $0\not\in T$, $-3\not\in T$, $\frac{3}{4}\not\in T$
				\end{itemize}
		\end{frame}
\end{document}
