\documentclass{article}
\usepackage{graphicx} % Required for inserting images
\usepackage[margin=1.0in]{geometry}
\usepackage{amsfonts}
\usepackage{amsmath}
\usepackage{amssymb}
\usepackage{amsthm}
\usepackage{mathtools}
\usepackage{centernot}

\newcommand{\R}{\mathbb{R}}
\newcommand{\Q}{\mathbb{Q}}
\newcommand{\N}{\mathbb{N}}
\newcommand{\C}{\mathbb{C}}

\theoremstyle{plain}
\newtheorem{thrm}{Theorem}
\newtheorem{lem}{Lemma}
\newtheorem{prop}{Proposition}
\newtheorem{cor}{Corollary}
\theoremstyle{definition}
\newtheorem{defn}{Definition}
\newtheorem{examp}{Example}


\title{Topology Presentation Notes}
\author{Nilay Tripathi }
\date{December 4, 2023}

\begin{document}

    \maketitle
    \tableofcontents
		
		\section{Metric Space Preliminaries}
		\begin{itemize}
				\item Sequences and Cauchy sequences 
						\begin{itemize}
								\item A sequence converges to a point $x_0\in X$ if 
										\begin{equation*}
												\forall \varepsilon > 0, \exists N_{\varepsilon} \in \N : n\geq N_{\varepsilon} \implies d(x_n, x) < \varepsilon
										\end{equation*}

								\item A sequence is Cauchy if 
										\begin{equation*}
												\forall \varepsilon > 0, \exists N_{\varepsilon} \in \N : m,n \geq N_{\varepsilon} \implies d(x_n, x_m) < \varepsilon
										\end{equation*}

								\item Every convergent sequence is Cauchy. PROOF: use triangle inequality with $\varepsilon / 2$ argument. 
						\end{itemize}

				\item Complete metric space: every Cauchy sequence in $X$ converges to a limit in $X$. 
						\begin{itemize}
								\item The metric space $\Q$ is not complete. PROOF: consider the sequence of rational approximations to any irrational number. 
								\item Discrete spaces are complete. PROOF: every Cauchy sequence is eventually constant. 
								\item Briefly comment about equivalent formulas of completeness in $\R$ (i.e. least upper bound property, montone convergence property, etc.). Cauchy completeness is the most general definition and works in all metric spaces.
						\end{itemize}
		\end{itemize}

		\subsection{Function Space}
		We define the \textbf{function space}, $C[a,b]$, to be all continuous functions from $[a,b]$ to $\R$. That is 
		\begin{equation*}
				C[a,b] = \{f : [a,b] \to \R \mid f \text{ is continuous} \}
		\end{equation*}
		We define a metric on the function space as follows: for $f,g\in C[a,b]$
		\begin{equation*}
				d(f, g) = \max_{t\in [a,b]} |f(t) - g(t)|
		\end{equation*}
		\begin{itemize}
				\item The metric is well defined (i.e. is finite). PROOF: $f-g$ is continuous and $[a,b]$ is compact. EVT implies existence of a maximum (and minimum)
		\end{itemize}

		\section{Vector Space Preliminaries}
		\begin{itemize}
				\item A vector space $V$ over a field $\mathbb{F}$ has two operations: vector addition and scalar multiplication where 
						\begin{itemize}
								\item Vector addition is an abelian group 
								\item Scalar multiplicatin satisfies: $1v = v$, $a(bv) = (ab)v$, and two distributive laws: scalar multiplication distributes over vector addition and field addition. 
						\end{itemize}

				\item Linearly independent sets: a (finite) set $V$ is linearly independent if for scalars $c_i$ and vectors $v_i$ 
						\begin{equation*}
								c_1v_1 + c_2v_2 + \cdots + c_nv_n = 0 \implies c_1 = c_2 = \cdots = c_n = 0
						\end{equation*}
						Infinite sets are linearly independent if all of its finite subsets are L.I. 

				\item If $V$ is a V.S. over $\mathbb{F}$ and $S \subseteq V$ is finite, then $\mathrm{span}\ S$ is defined as 
						\begin{equation*}
								\mathrm{span}\ S = \{c_1v_1 + \cdots + c_nv_n : c_i \in \mathbb{F}, v_i \in S\}
						\end{equation*}
						The span of an infinite set is the union of the span of all its finite subsets. 

				\item If $E \subseteq X$ is a subspace, then a set $S$ is a spanning set if $\mathrm{span}\ S = E$. 

				\item A basis is a linearly independent generating set. 
						\begin{itemize}
								\item It is the smallest generating set and the largest L.I. set (in a f.d. V.S.) 
								\item Every vector has a unique representation in a basis
								\item The dimension of a V.S. is the size of its basis (either finite or infinite)
								\item Every V.S. has a basis. For f.d. spaces, all bases have the same size 
						\end{itemize}
		\end{itemize}

		\subsection{Function Spaces}
		We turn the function space $C[a,b]$ into a vector space over $\R$. For $f,g\in C[a,b]$ and $\alpha\in \R$, define 
		\begin{align*}
				(f + g)(t) = f(t) + g(t) \\ 
				(\alpha f)(t) = \alpha f(t)
		\end{align*}
		The additive identity is the zero function. It is also an infinite dimensional V.S. 

		\subsection{Linear Maps}
		Linear maps preserve linear combos. So $T: X\to Y$ is linear if 
		\begin{equation*}
				T(c_1v_1 + c_2v_2 + \cdots c_nv_n) = c_1Tv_1 + c_2Tv_2 + \cdots + c_nTv_n
		\end{equation*}
		Notably, linear maps send the identity of $X$ to the identity of $Y$ (i.e. $T0 = 0$)
		\begin{itemize}
				\item The differentiation operator on $C[a,b]$, $Df = f'$, is linear. PROOF: derivative rules from calculus.
		\end{itemize}

		\section{Normed Spaces}
		\begin{itemize}
				\item A norm on a V.S. $V$ generalizes the length of a vector. It satisfies these axioms 
						\begin{enumerate}
								\item $\|x\| \geq 0$ with $\|x\| = 0 \Longleftrightarrow x = 0$
								\item $\|\alpha x\| = |\alpha| \|x\|$ (norm only depends on direction) 
								\item $\|x + y\| \leq \|x\| + \|y\|$ (satisfies triangle inequality)
						\end{enumerate}
						We use the word ``norm'' to mean both the value $\|x\|$ and the function $x \mapsto \|x\|$. 

				\item Normed space $\implies$ metric space 
						\begin{itemize}
								\item Define the metric as $d(x,y) = \|x-y\|$
						\end{itemize}
		\end{itemize}

		\subsection{Examples}
		\begin{itemize}
				\item The $L^p$-norms on $\R^n$ are defined by 
						\begin{equation*}
								\|x\|_p = \left[ \sum_{i=1}^{n} |x_i|^p \right]^{1/p}
						\end{equation*}
						\begin{itemize}
								\item The $L^p$-norm induces the $L^p$-metric on $\R^n$.
								\item If $p = 2$, this is the usual notion of length/distance on $\R^2$.
						\end{itemize}

				\item Consider the function space $C[a,b]$. Define a norm on this space as 
						\begin{equation*}
								\|f\| = \max_{t\in [a,b]} |f(t)|
						\end{equation*}

				\item Metric space $\centernot\implies$ normed space: the discrete metric on $\R$ is not induced by any norm. 
		\end{itemize}

		\subsection{Banach Spaces}
		A \textbf{Banach space} is any normed space where the norm induces a complete metric space. 
		\begin{itemize}
				\item $\R^n$ is a Banach space (when we consider its usual Euclidean norm) 
				\item The function space $C[a,b]$ is a Banach space, under its usual metric. 
				\item Not all normed spaces are Banach spaces. 
						\begin{itemize}
								\item Define a different norm on $C[a,b]$ as: 
										\begin{equation*}
												\|f\| = \int_0^1 |f(t)|\ dt 
										\end{equation*}
										The metric induced by the norm is 
										\begin{equation*}
												d(f,g) = \int_0^1 |f(t) - g(t)| \ dt
										\end{equation*}
										Under this metric $C[a,b]$ is not a complete space.
						\end{itemize}
		\end{itemize}

		\section{Inner Product Spaces}
		An \textbf{inner product} satisfies the following axioms 
		\begin{enumerate}
				\item $\langle x + y, z \rangle = \langle x, y \rangle + \langle y, z \rangle$ 
				\item $\langle \alpha x, y \rangle = \alpha \langle x, y \rangle$
				\item $\langle x, y \rangle = \overline{\langle y, x \rangle}$ 
				\item $\langle x, x \rangle \geq 0$ with $\langle x, x \rangle = 0 \iff x = 0$
		\end{enumerate}
		If the V.S. is over $\R$, axiom 3 becomes $\langle x, y \rangle = \langle y, x \rangle$.
		\begin{itemize}
				\item We have inner product space $\implies$ normed space. A norm may be defined as 
						\begin{equation*}
								\|x\| = \sqrt{\langle x, x\rangle}
						\end{equation*}
				\item However, the reverse implication is NOT true. EXAMPLE: $L^p$-norms when $p\neq 2$. 
		\end{itemize}

		\subsection{Examples}
		\begin{itemize}
				\item The standard inner product on $\R^n$ is 
						\begin{equation*}
								\langle x, y \rangle = \sum_{i=1}^{n} x_i y_i
						\end{equation*}
						It is better known as the dot product. It induces the Euclidean norm on $\R^n$. 

				\item If $X = \C^n$ instead, then the standard inner product becomes 
						\begin{equation*}
								\langle x, y \rangle = \sum_{i=1}^{n} x_i \overline{y_i}
						\end{equation*}
		\end{itemize}

		\subsection{Hilbert Space}
		A \textbf{Hilbert space} is an I.P.S. where the norm induced by the inner product induces a complete metric space. 
		\begin{itemize}
				\item Hilbert space $\implies$ Banach space 
				\item Converse not true. EXAMPLE: the max norm on $C[a,b]$ is not given by any inner product, so it cannot be a Hilbert space
				\item $\R^n$, with the Euclidean norm/standard inner product is a Hilbert space. 
		\end{itemize}

		\section{Topological Vector Spaces}
		We now combine the notions of vector spaces and topological spaces. A \textbf{topological vector space} is a vector space with a topology such that 
		\begin{enumerate}
				\item All one point sets in closed 
				\item The vector space operations $+$ and $\cdot$ are continuous
		\end{enumerate}

		\subsection{Continuous \& Bounded Linear Map}
		\begin{itemize}
				\item We say a linear map $T: X\to Y$ is continuous at $x_0\in X$ if for every $\varepsilon > 0$, there is a $\delta > 0$ such that 
				\begin{equation*}
						\|x - x_0\| < \delta \implies \|Tx - Tx_0\| < \varepsilon
				\end{equation*}
				If $T$ is continuous at all points in $X$, then $T$ is continuous. 

				\item We say a linear map $T: X\to Y$ is bounded if there exists $M\in\R$ such that 
						\begin{equation*}
								\|Tx\| \leq M\|x\|
						\end{equation*}

				\item We define the norm of an operator as follows 
						\begin{equation*}
								\|T\| = \sup_{\substack{x\in X \\ x \neq 0}} \frac{\|Tx\|}{\|x\|} = \sup_{\substack{x\in X \\ \|x\| = 1}} \|Tx\|
						\end{equation*}
						This gives us that for any bounded operator: $\|Tx\| \leq \|T\|\|x\|$. 
						\begin{itemize}
								\item The zero operator has norm 0. PROOF: 
										\begin{equation*}
												\frac{\|Tx\|}{\|x\|} = \frac{\|0\|}{\|x\|} = 0
										\end{equation*}
								\item The identity operator has norm 1. PROOF: 
										\begin{equation*}
												\frac{\|Tx\|}{\|x\|} = \frac{\|x\|}{\|x\|} = 1
										\end{equation*}
								\item The differentiation operator $D(f) = f'$ is unbounded. PROOF: consider polynomials on $[0, 1]$. If $x_n(t) = t^n$, then 
										\begin{equation*}
												\|Tx_n\| = \|nt^{n-1}\| = n
										\end{equation*}
						\end{itemize}
		\end{itemize}

		\subsection{Continuous $\iff$ Bounded} 
		We prove the following result. 
		\begin{thrm}[]
				A linear operator $T: X\to Y$, between normed spaces is continuous iff it is bounded. 
		\end{thrm}
		\begin{proof}
				Assume $T$ is bounded. Then, we have $\|T\| < \infty$. Let $\varepsilon > 0$ and $x_0\in X$ be arbitrary. Let 
				\begin{equation*}
						\delta = \frac{\varepsilon}{\|T\|}
				\end{equation*}
				and assume $\|x - x_0\| < \delta$. Then, we have 
				\begin{align*}
						\|Tx - Tx_0\| &= \|T(x - x_0)\| \\ 
													&\leq \|T\|\|x - x_0\| \\ 
													&< \|T\| \cdot \frac{\varepsilon}{\|T\|} \\ 
													&= \varepsilon
				\end{align*}
				Hence, $T$ is continuous as it is continuous at all points. \par 

				Conversely, assume $T$ is continuous. Then, $T$ is continuous at $x = 0$. So, there is a $\delta > 0$ such that 
				\begin{equation*}
						\|x\| < \delta \implies \|Tx\| < 1
				\end{equation*}
				Rewrite $x$ as 
				\begin{equation*}
						x = \frac{2x \delta}{\|x\|} \cdot \frac{\|x\|}{2 \delta}
				\end{equation*}
				So 
				\begin{align*}
						Tx &= T \left( \frac{2x \delta}{\|x\|} \cdot \frac{\|x\|}{2 \delta} \right) \\ 
							 &= \frac{2\|x\|}{\delta} T \left( \frac{x}{\|x\|} \cdot \frac{\delta}{2} \right)
				\end{align*}
				Put $v = \frac{x}{\|x\|} \cdot \frac{\delta}{2}$ and note that 
				\begin{equation*}
						\|v\| = \frac{\delta}{2} \left\| \frac{x}{\|x\|} \right\| = \frac{\delta}{2} < \delta
				\end{equation*}
				So $\|Tv\| < 1$. Hence, we see that 
				\begin{equation*}
						\|Tx\| < \frac{2}{\delta}\|x\|
				\end{equation*}
				which shows that $T$ is bounded. 
		\end{proof}

		\section{Closed Graph Theorem} 

		\subsection{Open Mapping Theorem} 
		We state the open mapping theorem, and its corollary, the bounded inverse theorem. 
		\begin{itemize}
				\item An open mapping sends open sets in the domain to open sets in the range i.e. the image of open sets is open 
				\item Open Mapping Thrm: every surjective bounded linear operator between Banach spaces is an open mapping 
						\begin{itemize}
								\item IDEA OF PROOF: similar to the ``neighborhood trick'': show that for an open set $U$, any point in $T(U)$ contains an open set contained in $T(U)$. 
								\item However, the details are complicated since we are in an arbitrary space 
						\end{itemize}
				\item Bounded Inverse Thrm: if $T$ is a bijective, continuous linear operator, then $T^{-1}$ is bounded. 
						\begin{itemize}
								\item IDEA OF PROOF: show $T^{-1}$ is continuous. Indeed, $T: X\to Y$ maps open sets to open sets. But this is roughly the same as the preimage of open sets in $X$ being open under $T^{-1}$. Thus, $T^{-1}$ is continuous and, therefore, bounded. 
						\end{itemize}
		\end{itemize}

		\subsection{Norm On Product Space}
		If $X,Y$ are Banach spaces, we define a norm on $X\times Y$ as follows 
		\begin{equation*}
				\|(x,y)\| = \|x\| + \|y\|
		\end{equation*}
		We remark: there are other norms on the product space (e.g. the ``max'' norm), but the closed graph theorem will go through for all of them. I think this is the easiest one. 
		\begin{thrm}[]
				If $X,Y$ are Banach spaces, then so is $X\times Y$. 
		\end{thrm}
		\begin{proof}[Proof Sketch]
				Let $(z_n)$ be an arbitrary Cauchy sequence. W.T.S. $(z_n)$ converges
				\begin{itemize}
						\item From definition of the product norm, we have $(x_n)$ and $(y_n)$ are Cauchy
						\item The two Cauchy sequences converge. Then, $(z_n)$ converges to the ordered pair (limit of $x_n$, limit of $y_n$) 
						\item To show this, use a $\varepsilon/2$ argument. 
				\end{itemize}
		\end{proof}

		\subsection{Proof of Closed Graph Theorem}
		A linear operator $T: X\to Y$ if the graph of $T$ 
		\begin{equation*}
				G_T = \{(x, Tx) : x\in X\}
		\end{equation*}
		is a closed set in $X\times Y$. 
		\begin{thrm}[Closed Graph Thrm]
				Suppose $T: D\to Y$ is a closed linear operator. Then, $T$ is bounded if $D$ is closed. 
		\end{thrm}
		\begin{proof}[Proof Idea]
				\begin{itemize}
						\item Define a linear map $P: G_T \to D$ as 
								\begin{equation*}
										P(x, Tx) = x
								\end{equation*}
						\item $P$ is linear since 
								\begin{equation*}
										P(x+cy, T(x+cy)) = P(x+cy, Tx+cTy) = P[(x, Tx) + c(y, Ty)] 
								\end{equation*}
						\item $P$ is bounded since 
								\begin{equation*}
										\|P(x, Tx)\| = \|x\| \leq \|x\| + \|Tx\| = \|(x, Tx)\|
								\end{equation*}
						\item $P$ is bijective. 
						\item $G_T$ is a closed subset of $X\times Y$, a complete space $\implies G_T$ is closed. 
						\item Same argument shows $D$ is a complete space. 
						\item Invoke the bounded inverse theorem to get that $P^{-1}$ is bounded. 
						\item The proof is finished by noting that 
								\begin{align*}
										\|Tx\| &\leq \|x\| + \|Tx\| \\ 
													 &= \|(x, Tx)\| \\ 
													 &= \|P^{-1}(x)\| \\ 
													 &\leq M\|x\|
								\end{align*}
								So $T$ is bounded, as desired. 
				\end{itemize}
		\end{proof}
\end{document}
